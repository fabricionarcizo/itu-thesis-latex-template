%
% MIT License
%
% Copyright (c) 2024 Fabricio Batista Narcizo.
%
% Permission is hereby granted, free of charge, to any person obtaining a copy
% of this software and associated documentation files (the "Software"), to deal
% in the Software without restriction, including without limitation the rights
% to use, copy, modify, merge, publish, distribute, sublicense, and/or sell
% copies of the Software, and to permit persons to whom the Software is
% furnished to do so, subject to the following conditions:
%
% The above copyright notice and this permission notice shall be included in
% all copies or substantial portions of the Software.
%
% THE SOFTWARE IS PROVIDED "AS IS", WITHOUT WARRANTY OF ANY KIND, EXPRESS OR
% IMPLIED, INCLUDING BUT NOT LIMITED TO THE WARRANTIES OF MERCHANTABILITY,
% FITNESS FOR A PARTICULAR PURPOSE AND NONINFRINGEMENT. IN NO EVENT SHALL THE
% AUTHORS OR COPYRIGHT HOLDERS BE LIABLE FOR ANY CLAIM, DAMAGES OR OTHER
% LIABILITY, WHETHER IN AN ACTION OF CONTRACT, TORT OR OTHERWISE, ARISING FROM,
% OUT OF OR IN CONNECTION WITH THE SOFTWARE OR THE USE OR OTHER DEALINGS IN THE
% SOFTWARE.
%

% Source Code Appendix.
\chapter{Source Code}\label{app:source-code}
\lettrine[lines=3]{P}{lease,} avoid to add source-codes directly in your thesis chapters. Instead, add them as an appendix or in a separate document. This approach ensures that the document remains clean and easy to read. Below is an example of how to include a Python script in the document. See Listing~\ref{lst:cnn_2} for the source code.
\lstinputlisting[
    language=Python,
    belowcaptionskip=15pt,
    label={lst:cnn_2},
    caption={Convolutional neural network implemented in Python using TensorFlow and Keras.}]{codes/cnn_model.py}
