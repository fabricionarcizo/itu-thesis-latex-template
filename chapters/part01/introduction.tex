%
% MIT License
%
% Copyright (c) 2024 Fabricio Batista Narcizo.
%
% Permission is hereby granted, free of charge, to any person obtaining a copy
% of this software and associated documentation files (the "Software"), to deal
% in the Software without restriction, including without limitation the rights
% to use, copy, modify, merge, publish, distribute, sublicense, and/or sell
% copies of the Software, and to permit persons to whom the Software is
% furnished to do so, subject to the following conditions:
%
% The above copyright notice and this permission notice shall be included in
% all copies or substantial portions of the Software.
%
% THE SOFTWARE IS PROVIDED "AS IS", WITHOUT WARRANTY OF ANY KIND, EXPRESS OR
% IMPLIED, INCLUDING BUT NOT LIMITED TO THE WARRANTIES OF MERCHANTABILITY,
% FITNESS FOR A PARTICULAR PURPOSE AND NONINFRINGEMENT. IN NO EVENT SHALL THE
% AUTHORS OR COPYRIGHT HOLDERS BE LIABLE FOR ANY CLAIM, DAMAGES OR OTHER
% LIABILITY, WHETHER IN AN ACTION OF CONTRACT, TORT OR OTHERWISE, ARISING FROM,
% OUT OF OR IN CONNECTION WITH THE SOFTWARE OR THE USE OR OTHER DEALINGS IN THE
% SOFTWARE.
%

% Introduction Chapter.
\chapter{Introduction}\label{chap:introduction}
\lettrine[lines=3]{T}{his} \LaTeX~template provides a structured and professional foundation for the bachelor's, master's, and Ph.D. students from the \href{http://itu.dk}{\acrfull{itu}}, ensuring consistency, clarity, and ease of use as they develop their theses. The students can use this template to guide them in organizing their work effectively, allowing them to focus on the content while adhering to academic formatting standards.

The introduction chapter is where you set the stage for your thesis. It provides the reader with the necessary background, clearly defines the problem, and outlines the goals and significance of your study. This chapter serves as a roadmap for the rest of the thesis, giving the audience a clear understanding of what to expect and why your work matters.

\section{Background and Context}\label{sec:background-context}
This section introduces the reader to the broader context of your research area. You should:
\begin{enumerate}
	\item \textbf{Describe the Field:} Provide an overview of your work's domain or discipline. Discuss recent trends or developments that make your research relevant.
	\item \textbf{Highlight the Motivation:} Explain what motivated you to pursue this topic. This might include societal challenges, technological advancements, or gaps in existing knowledge.
\end{enumerate}

Keep this section concise but informative. The goal is to give the reader enough context to appreciate the importance of the problem you aim to solve.

\section{Problem Statement}\label{sec:problem-statement}
Here, you define the specific problem your research addresses. This section should include:
\begin{enumerate}
	\item \textbf{Current Challenges:} Summarize the key issues or limitations in the current state of knowledge or practice.
    \item \textbf{Focus of the Study:} Clearly articulate the specific problem you aim to solve, ensuring it is well-defined and measurable.
\end{enumerate}

A well-written problem statement helps the reader immediately grasp your thesis's core issue.

\section{Objectives of the Study}\label{sec:objectives}
In this section, clearly outline the goals of your research. These objectives should be specific, measurable, and aligned with the problem statement. Use bullet points or numbered lists to make them easy to read. For example:
\begin{itemize}
	\item \textit{To analyze\dots}
    \item \textit{To develop\dots}
    \item \textit{To evaluate\dots}
\end{itemize}

Ensure your objectives are realistic and achievable within the scope of your thesis.

\section{Research Questions and Hypotheses}\label{sec:research-questions}
Here, you present the research questions that guide your study. These questions should be directly related to your objectives and problem statement. If applicable, state your hypotheses, explaining the assumptions or predictions your study will test. For example:
\begin{itemize}
	\item \textbf{Research Question:} How does [variable] influence [outcome]?
    \item \textbf{Hypothesis:} It is hypothesized that [specific prediction].
\end{itemize}

This section demonstrates the focus and clarity of your study.

\section{Scope and Limitations}\label{sec:scope-limitations}
This section defines the boundaries of your research and acknowledges its limitations. Include:
\begin{enumerate}
	\item \textbf{Scope:} Explain what your study covers (e.g., data, methods, population).
	\item \textbf{Limitations:} Be transparent about time, resources, or data availability constraints that may affect your results.
\end{enumerate}

Being upfront about the limitations, adds credibility to your work and sets realistic expectations for the reader.

\section{Significance of the Study}\label{sec:significance}
Discuss why your research is essential and its contribution to the field. Address questions like:
\begin{itemize}
	\item How does this work fill existing gaps in knowledge?
	\item What are the practical or theoretical implications of your findings?
\end{itemize}

Convince the reader that your study has meaningful value and relevance.

\section{Structure of the Thesis}\label{sec:struture-thesis}
This final section provides a brief overview of the chapters to follow. Write 1-2 sentences summarizing the content of each chapter. For example:

This thesis has six chapters, each addressing specific aspects of the research topic. The introduction composes Chapter~\ref{chap:introduction} and sets the scene for the presented research project.

Chapter~\ref{chap:review} provides a comprehensive overview of the relevant literature and existing research in the field. It critically evaluates previous works, identifies gaps in the knowledge, and establishes the theoretical foundation for the study.

Chapter~\ref{chap:methodology} describes the research approach and methods to achieve the study's objectives. It includes details on data collection, experimental design, implementation, and evaluation metrics. It also discusses ethical considerations and limitations of the methodology.

Chapter~\ref{chap:results} presents the research findings clearly and systematically. It includes data visualizations, tables, and detailed analyses to interpret the results in the context of the research questions.

Chapter~\ref{chap:discussion} provides a deeper interpretation of the results, comparing them with existing studies. It discusses the findings' implications, the study's strengths and limitations, and potential areas for future research.

Chapter~\ref{chap:conclusions} is the last chapter, and it summarizes the key findings, highlights the study's contributions, and provides actionable recommendations. It concludes with final reflections on the research and its broader impact.
