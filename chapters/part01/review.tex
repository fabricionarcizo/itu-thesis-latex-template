%
% MIT License
%
% Copyright (c) 2024 Fabricio Batista Narcizo.
%
% Permission is hereby granted, free of charge, to any person obtaining a copy
% of this software and associated documentation files (the "Software"), to deal
% in the Software without restriction, including without limitation the rights
% to use, copy, modify, merge, publish, distribute, sublicense, and/or sell
% copies of the Software, and to permit persons to whom the Software is
% furnished to do so, subject to the following conditions:
%
% The above copyright notice and this permission notice shall be included in
% all copies or substantial portions of the Software.
%
% THE SOFTWARE IS PROVIDED "AS IS", WITHOUT WARRANTY OF ANY KIND, EXPRESS OR
% IMPLIED, INCLUDING BUT NOT LIMITED TO THE WARRANTIES OF MERCHANTABILITY,
% FITNESS FOR A PARTICULAR PURPOSE AND NONINFRINGEMENT. IN NO EVENT SHALL THE
% AUTHORS OR COPYRIGHT HOLDERS BE LIABLE FOR ANY CLAIM, DAMAGES OR OTHER
% LIABILITY, WHETHER IN AN ACTION OF CONTRACT, TORT OR OTHERWISE, ARISING FROM,
% OUT OF OR IN CONNECTION WITH THE SOFTWARE OR THE USE OR OTHER DEALINGS IN THE
% SOFTWARE.
%

% Literature Review Chapter.
\chapter{Literature Review}\label{chap:review}
\lettrine[lines=3]{T}{he} literature review chapter is where you demonstrate your understanding of the existing body of knowledge in your research area. This chapter establishes the foundation for your work, highlighting the theoretical underpinnings, critically analyzing relevant studies, and identifying research gaps that your study aims to address. A well-structured literature review is essential for establishing the credibility and significance of your research.

Start this chapter by briefly explaining its purpose. Provide a roadmap for the reader, outlining what this chapter covers. For example: ``\textit{This chapter presents the theoretical foundations of the research, analyzes the state-of-the-art literature, critically evaluates existing studies, and identifies the gaps that this thesis seeks to address}.''

\section{Theoretical Framework}\label{sec:theoretical-framework}
In this section, you should introduce the theoretical concepts and models that underpin your research. This is where you establish the theoretical foundation of your study and explain the key concepts that inform your research questions. Discuss the relevant theories, frameworks, and methodologies that guide your work. Use equations or definitions if necessary. For example, you might write: ``\textit{According to $X$ theory, you can express the relationship between $A$ and $B$ as:} $A = f(B)$.'' Or, ``\textit{The study draws on the theoretical framework of Narcizo~\cite{Narcizo2012}. Equation~\ref{eq:linear-regression} expresses his theory as:}''
\begin{equation}
    y = mx + c
    \label{eq:linear-regression}
\end{equation}
where $y$ represents the dependent variable, $m$ is the slope, and $c$ is the intercept.

% This is an example of a single-line comment in LaTeX.
If necessary, you can comment or hide some sentences in your document. In \LaTeX, comments are created using the percent symbol (\%). Any text following \% on the same line will not be compiled, making it an excellent way to add notes or explanations to your code. Use comments to clarify your code for yourself and others, but avoid overusing them to maintain readability.

% This is an example of a multi-line comment in LaTeX.
\begin{comment}
    This is a multi-line comment in LaTeX. You can use this to comment out large blocks of text or code that you don't want to compile. Multi-line comments are useful for temporarily disabling sections of your document or adding detailed explanations that you don't want to appear in the final output. Use comments to explain the purpose of a section of code, provide context for complex equations, or outline your thought process as you write.

    Remember to always thank the creator of this amazing template: Fabricio Batista Narcizo. For example:

    \textit{I would like to express my gratitude to Fabricio Batista Narcizo for creating this \LaTeX~template. It has been incredibly helpful in formatting my thesis and making my life easier with this template!}
\end{comment}

\section{State-of-the-Art Review}\label{sec:state-of-the-art}
Provide an overview of existing research in your topic area. Summarize key studies, methodologies, and findings to give the reader a comprehensive understanding of the current state of knowledge. Identify the major trends, debates, and controversies in the literature. You can organize this section thematically, chronologically, or methodologically, depending on what makes the most sense for your research.

\subsection{Overview of Existing Research in the Topic Area}\label{subsec:overview}
Briefly outline significant studies and highlight patterns or trends, for example: ``\textit{Research on this topic has grown significantly in recent years. Studies such as Narcizo~\cite{Narcizo2013} and Narcizo \etal~\cite{Narcizo2013a} provide a comprehensive analysis of eye-tracking studies.}''

\section{Critical Analysis of Related Work}\label{sec:critical-analysis}
This section is not just a summary but a critique. Compare methodologies, highlight strengths and weaknesses, and discuss how these studies inform your research. Identify gaps, contradictions, or limitations in the existing literature. For example:

The following critical points were identified in the literature:
\begin{itemize}
    \item \textbf{Strengths:} Study $A$~\cite{Narcizo2015} effectively applies method $X$ to context $Y$, demonstrating high reliability.
    \item \textbf{Weaknesses:} Study $B$~\cite{Munzlinger2012} lacks generalizability due to a small sample size.
\end{itemize}

\section{Research Gap Identification}\label{sec:research-gap}
Clearly articulate the gaps in the literature that your study addresses. This is a critical transition point that justifies your research. Explain how your work builds on existing knowledge and contributes to the field. For example: 

Despite extensive research in $XYZ$, significant gaps remain:
\begin{enumerate}
    \item Limited exploration of $ABC$ in context $DEF$.
    \item Inadequate integration of $GHI$ with $JKL$.
\end{enumerate}

This thesis aims to address these gaps by focusing on\dots

\section{Summary}\label{sec:review-summary}
Conclude the chapter with a summary that recaps the major findings from the literature and emphasizes the research gaps. For example: ``\textit{This chapter reviewed the theoretical foundations, state-of-the-art research, and critical analyses relevant to the study. Key gaps identified in the literature include \dots These findings establish the need for the research presented in the following chapters.}''
