%
% MIT License
%
% Copyright (c) 2024 Fabricio Batista Narcizo.
%
% Permission is hereby granted, free of charge, to any person obtaining a copy
% of this software and associated documentation files (the "Software"), to deal
% in the Software without restriction, including without limitation the rights
% to use, copy, modify, merge, publish, distribute, sublicense, and/or sell
% copies of the Software, and to permit persons to whom the Software is
% furnished to do so, subject to the following conditions:
%
% The above copyright notice and this permission notice shall be included in
% all copies or substantial portions of the Software.
%
% THE SOFTWARE IS PROVIDED "AS IS", WITHOUT WARRANTY OF ANY KIND, EXPRESS OR
% IMPLIED, INCLUDING BUT NOT LIMITED TO THE WARRANTIES OF MERCHANTABILITY,
% FITNESS FOR A PARTICULAR PURPOSE AND NONINFRINGEMENT. IN NO EVENT SHALL THE
% AUTHORS OR COPYRIGHT HOLDERS BE LIABLE FOR ANY CLAIM, DAMAGES OR OTHER
% LIABILITY, WHETHER IN AN ACTION OF CONTRACT, TORT OR OTHERWISE, ARISING FROM,
% OUT OF OR IN CONNECTION WITH THE SOFTWARE OR THE USE OR OTHER DEALINGS IN THE
% SOFTWARE.
%

% Discussion Chapter.
\chapter{Discussion}\label{chap:discussion}
\lettrine[lines=3]{T}{he} discussion chapter is where you interpret your findings in depth, placing them in the context of existing research and explaining their significance. This chapter is an opportunity to critically evaluate your work, address its limitations, and propose directions for future research. Start the chapter by summarizing the key findings from your results chapter. Highlight how these findings answer the research questions or address the hypotheses. For example: 

This chapter provides a concise summary of the primary outcomes of this research, emphasizing their relevance to the stated objectives and hypotheses. The analysis revealed several significant insights:
\begin{itemize}
    \item Model $P_e^{s*}$ achieved the highest accuracy (63\%) among all tested models.
    \item The proposed methodology improved performance metrics compared to existing techniques (see Table~\ref{tab:real-gaussian-distribution}).
    \item This supports the hypothesis $H_1: \mu_1 \neq \mu_2$.
\end{itemize}

\section{Comparison with Existing Work}\label{sec:comparison-existing-work}
Compare your findings with those from the literature. Highlight similarities, differences, and how your work contributes to the existing body of knowledge. For example:

The results align with prior studies, such as Melo et al.~\cite{Melo2017}, who observed similar trends in model performance. However, unlike Narcizo~\cite{Narcizo2013}, this study demonstrates that \dots As illustrated in Figure~\ref{fig:real-data-histogram}, Model $P_e^{s*}$ outperformed benchmarks by a significant margin.

\section{Implications of Findings}\label{sec:implications-findings}
Discuss the practical, theoretical, or methodological implications of your research. Explain how your findings can be applied in real-world scenarios or how they advance the field. For example:

The findings have several important implications:
\begin{enumerate}
    \item \textbf{Practical Implications:} The proposed model can be integrated into real-time systems for $XYZ$ applications.
    \item \textbf{Theoretical Implications:} This study provides evidence supporting the hypothesis that \dots
\end{enumerate}

\section{Strengths and Limitations}\label{sec:strengths-limitations}
Critically evaluate your work, acknowledging its strengths while being transparent about its limitations. This section adds credibility by showing awareness of potential weaknesses. For example:

This research has several strengths:
\begin{itemize}
    \item Rigorous evaluation using state-of-the-art benchmarks.
    \item Use of diverse datasets to ensure generalizability.
\end{itemize}

However, there are limitations:
\begin{itemize}
    \item Limited dataset size for certain experiments, which may affect statistical power.
    \item The scope of the study excludes \dots
\end{itemize}

\section{Future Research Directions}\label{sec:future-research}
Propose potential directions for extending your research. This demonstrates a forward-thinking approach and contributes to the ongoing development of the field. For example:

Future research could explore the following directions:
\begin{enumerate}
    \item Investigating the scalability of the proposed methodology to larger datasets.
    \item Extending the approach to incorporate deep learning techniques for \dots
    \item Conducting cross-domain studies to validate the generalizability of findings.
\end{enumerate}
